\documentclass[onecolumn]{article}
\begin{document}
\title{CS 491: Homework 1, Test Case Report}
\author{Armando Diaz Tolentino (adt), adiazt2@uic.edu }
\maketitle

\begin{abstract}
For this assignment, I have re-implemented the standard C libraries for memory and string handling. 
Functions like memcpy(), memset(), strcat(), and strcmp(). For more information
refer to the README file and doxygen documentation provided.
\end{abstract}

\section{Provided Test Cases}
The test cases provided in the assignment handout are given in this section:
\subsection{Case 1}
Test Case :
\begin{verbatim}
void main(int argc, char **argv) {

	char* cat = "cat" ;
	char* dogGoesWoof ="dog\0goes\0woof" ;
	char str[100] ;
	memset( str, 0, 100 ) ;
	strcat( str, cat ) ;
	printf( "%s", str) ;	// test 1
	printf("\n") ;
	strcpy( str, dogGoesWoof ) ;
	printf( "%s", str) ;	// test 2
	printf("\n") ;
	strcpy( str+1, str ) ;
	memcpy( str, cat, 22 ) ;
	printf( "%s", str) ;	// test  3
}
\end{verbatim}


As expected, str prints correctly for the first two cases.
For the third case, we get a segmentation fault. This
occurs because of the command 
"strcpy( str+1, str ) ; "
My implementation of strcpy(), like the standard library 
implementations, fails in the case where the two input
strings overlap in memory.
\begin{verbatim}
cer0@skynet:~/Documents/courses/cs491/projects/hw1$ ./tst
cat
dog
Segmentation fault
\end{verbatim}

\subsection{Case 2}
\begin{verbatim}
void main(int argc, char **argv) {
	char* ptr = 0 ;
	*ptr = 'a' ;
}
\end{verbatim}


As expected a pointer to NULL, yields a run-time error.
This would actually have nothing to do with my code.
\begin{verbatim}
cer0@skynet:~/Documents/courses/cs491/projects/hw1$ ./tst
Segmentation fault
\end{verbatim}

\section{Additional Test Cases}
\subsection{More Args than Format Specifiers}

Here we test for giving printf() more arguments than
the format string specifies.
\begin{verbatim}
void main(int argc, char **argv) {

	char* name = "Ra's Al Ghul", *hero = "Batman" ;
	int someNum = 19872, otherNum = 45 ;
	long lNum = 998292 ;

	/* more arguments than format specifiers */
	printf( "%s\n", name, hero, someNum, otherNum, lNum ) ;
	printf( "%d\n", someNum, hero, name, otherNum, lNum ) ;
	printf( "%ld\n", lNum, hero) ;
}
\end{verbatim}


As expected additional arguments are simply not printed. 
No error occurs.
\begin{verbatim}
cer0@skynet:~/Documents/courses/cs491/projects/hw1$ ./tst
Ra's Al Ghul
19872
998292
\end{verbatim}

\subsection{More Format Specifiers than Args: Missing String }
Here we test behavior when more format flags are provided, than 
arguments to printf().
\begin{verbatim}
void main(int argc, char **argv) {
	char* hello = "Konnichiwa, minasan :)" ;
	printf("%s\n%s\n%s\n") ;
}
\end{verbatim}


In the case where a string is expected first, and no arguments are 
given, a NULL value is used for the string. This caused my 
ASSERT macro to be called, when printf() attempted to copy
the non-existent string to the memory buffer provided.
\begin{verbatim}
cer0@skynet:~/Documents/courses/cs491/projects/hw1$ ./tst

Assert failed: dest != NULL && src != NULL
 in file mem.c
 at line 109

Terminated
\end{verbatim}
This second case demonstrates what occurs when a string 
argument is missing, but another at least has been provided.
\begin{verbatim}
void main(int argc, char **argv) {
	char* hello = "Konnichiwa, minasan :)" ;
	printf("%s\n%s\n%s\n", hello ) ;
}
\end{verbatim}
In this case garbage values are printed out. 
We begin printing from a certain location in memory,
and continue until an NULL terminator is encountered. I am not certain
why in particular this happens. It is my understanding that
the va\_arg() function returns a 0 value, when no further 
arguments have been provided. In this case, we should get
an ASSERT failure as in the previous test case.

\begin{verbatim}
cer0@skynet:~/Documents/courses/cs491/projects/hw1$ ./tst
Konnichiwa, minasan :)
��t,�T$
       e3
������P���D���D���

\end{verbatim}
\subsection{More Format Specifiers than Args: Missing int }
Here we test the output of printf() when an integer value is 
missing. 
\begin{verbatim}
void main(int argc, char **argv) {
	char* hello = "Konnichiwa, minasan :)" ;
	printf("%s\n%d\n%d\n", hello, 4544) ;
}
\end{verbatim}

In the place of the given integer flag, printf() prints
a garbage value for the integer. Two test cases below
demonstrate that the integer isn't the same every time
we run our code.
\begin{verbatim}
cer0@skynet:~/Documents/courses/cs491/projects/hw1$ ./tst
Konnichiwa, minasan :)
4544
-216081116


cer0@skynet:~/Documents/courses/cs491/projects/hw1$ ./tst
Konnichiwa, minasan :)
4544
-217113308
\end{verbatim}

\subsection{ Parameter Type Mismatch : string as an int }
Here we test printf() behavior when a string is interpreted
as an integer type.
\begin{verbatim}
void main(int argc, char **argv) {
	char* iSay="Chunky Bacon ~===~ !!!!!!!" ;
	printf( "%x\n", iSay) ;
}
\end{verbatim}


printf() ends up printing the memory address that the 
string begins at. This is not surprising considering
basic knowledge of c-strings.
\begin{verbatim}
cer0@skynet:~/Documents/courses/cs491/projects/hw1$ ./tst
0x804955C
\end{verbatim}
\subsection{ Parameter Type Mismatch : int as a string }
Here we test what occurs when an integer is interpreted
as a string pointer.
\begin{verbatim}
void main(int argc, char **argv) {
	printf( "%s\n", 2343432) ;
}
\end{verbatim}


As expected an arbitrary int is interpreted as a memory
address which we don't in this case have access to. Thus
a segmentation fault occurs.
\begin{verbatim}
cer0@skynet:~/Documents/courses/cs491/projects/hw1$ ./tst
Segmentation fault
\end{verbatim}
\section{Conclusion}
The above represents only a sampling of tests done to verify 
the contents of hw1. The other library functions provided
were also extensively tested, but their test cases are not
provided here for brevity's sake.
\end{document}
